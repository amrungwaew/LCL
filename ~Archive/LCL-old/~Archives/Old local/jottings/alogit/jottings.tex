\documentclass[12pt]{article}


\usepackage[pdftex]{graphicx}
\usepackage[T1]{fontenc}
\usepackage[amsthm]{newtx}
\usepackage{bbm, bm}
\usepackage{setspace}
\usepackage{enumitem}
\usepackage{mathtools}

% Typsetting proofs
\theoremstyle{plain} 
\newtheorem{theorem}{Theorem}
\newtheorem{lemma}{Lemma}
\newtheorem{corollary}{Corollary}
\renewcommand{\qedsymbol}{\(\blacksquare\)}

% Neat tables
\usepackage[labelsep=endash,font=sc,justification=centering]{caption}
\usepackage[labelsep=period,font=it,labelformat=simple]{subcaption}
\usepackage{booktabs}
\usepackage{multirow}
\usepackage{makecell}
\usepackage{dcolumn}
\usepackage{siunitx}
\sisetup{
  detect-all,
  group-digits             = true,
  group-minimum-digits     = 5,
  group-separator          = {,},
  table-align-text-pre     = false,
  table-align-text-post    = false,
  input-signs              = + -,
  %input-symbols            = {*} {**} {***},
  input-open-uncertainty   = ,
  input-close-uncertainty  = ,
  retain-explicit-plus
}
\usepackage[flushleft]{threeparttable}
\usepackage{xpatch}

\usepackage[hidelinks, pdfstartview=]{hyperref}
\usepackage[noabbrev,capitalise]{cleveref}
\crefname{equation}{equation}{equations}
% \crefname{table}{Table}{Tables}
\crefname{appfig}{Appendix Figure}{Appendix Figures}
\crefname{apptab}{Appendix Table}{Appendix Tables}
% \crefname{equation}{}{}

\usepackage[margin=1in]{geometry}
\usepackage[bottom]{footmisc}
\usepackage{abstract}

\makeatletter
\chardef\TPT@@@asteriskcatcode=\catcode`*
\catcode`*=11
\xpatchcmd{\threeparttable}
{\TPT@hookin{tabular}}
{\TPT@hookin{tabular}\TPT@hookin{tabu}}
{}{}
\catcode`*=\TPT@@@asteriskcatcode
\makeatother
\makeatletter
\renewcommand\@biblabel[1]{}
\makeatother
\parskip 0cm
\parindent .5cm
\textwidth=6.67in
\oddsidemargin .0in
\topmargin -.5in
\textheight 9in

% Format section headings
% ======================
\usepackage{sectsty}
\subsectionfont{\large\normalfont\itshape}
\subsubsectionfont{\normalsize\normalfont\itshape}

\makeatletter
\renewcommand\thesubsection{\Alph{subsection}}
\renewcommand{\@seccntformat}[1]{\csname the#1\endcsname.\quad}
\renewcommand{\p@subsection}{\thesection} % for cross-refs to subsections
\makeatother
% =======================

% Format exhibit captions
% =======================
\renewcommand{\thefigure}{\textmd{\arabic{figure}}}
\renewcommand{\thetable}{\textmd{\arabic{table}}}

\renewcommand{\figurename}{{ \sc Figure}}
\renewcommand{\thesubfigure}{Panel \textmd{\Alph{subfigure}}}
\renewcommand{\tablename}{{ \sc Table}}
\renewcommand{\thesubtable}{Panel \Alph{subtable}}
% =======================

\newcolumntype{d}[1]{D{.}{.}{#1}}
\newcommand\mc[1]{\multicolumn{1}{c}{#1}}% <---
\def\sym#1{\ifmmode^{#1}\else\(^{#1}\)\fi}

\DeclareMathOperator*{\argmin}{arg\,min}
\DeclareMathOperator*{\argmax}{arg\,max}

\DeclareMathOperator{\wideargmin}{arg\,min}
\DeclareMathOperator{\wideargmax}{arg\,max}

\DeclareMathOperator{\widemin}{min}
\DeclareMathOperator{\widemax}{max}

\usepackage{float}

\renewcommand{\topfraction}{.85}
\renewcommand{\bottomfraction}{.7}
\renewcommand{\textfraction}{.15}
\renewcommand{\floatpagefraction}{.66}
\renewcommand{\dbltopfraction}{.66}
\renewcommand{\dblfloatpagefraction}{.66}
\setcounter{topnumber}{9}
\setcounter{bottomnumber}{9}
\setcounter{totalnumber}{20}
\setcounter{dbltopnumber}{9}

\raggedbottom
\singlespacing

\begin{document}

\title{Gradient of Within/Across Model}
\author{Andrew~Zeyveld}
\date{\today}
\maketitle
\onehalfspacing


\section{Likelihood}

To streamline notation, let the vector \(\bm{\vartheta} \coloneq ( \bar{\bm{\beta}}_i, \omega_{it}^d, \alpha_i) \) collect the taste and learning parameters associated with consumer \(i\). Likewise, the vector \(w_{jt} \coloneq (x_j, p_{jt}, \xi_{jt})\) collect the observable characteristics, price, and unobserved demand shock associated with good \(j\). Finally, denote the representative utility of good \(j\), namely \(u_{ijt} - \varepsilon_{ijt}\), by 
\begin{equation*}
  V(w_{jt}; \bm{\vartheta}_{it}) = x_j \bm{\beta}_{it} - \alpha_i p_{jt} + \xi_{jt}.
\end{equation*}

Then, by the (conditional) logit formula, the probability of good \(j\)'s being ordered by consumer \(i\) at time \(t\) is 
\begin{equation*}
  \operatorname{Pr} [ \text{order}_{ijt} = 1 \mid  \bm{\vartheta}_{it} ] \coloneq \frac{\exp \big(  V(w_{jt}; \bm{\vartheta}_{it}) \big)}{\sum_{j' \in \mathcal{J}_t} \big(  V(w_{j't}; \bm{\vartheta}_{it}) \big)}.
\end{equation*}

If the consumer's preferred good goes out of stock, the probability of her accepting good \(s\) as a substitute is 
\begin{equation*}
  \operatorname{Pr} [ \text{accept}_{ist} = 1 \mid \bm{\vartheta}_{it} ] \coloneq \frac{\exp \big(  V(w_{st}; \bm{\vartheta}_{it}) \big)}{1 + \exp \big(  V(w_{st}; \bm{\vartheta}_{it}) \big)}.
\end{equation*}


\section{Gradient}

I will now derive the gradient of the likelihood function. In deriving the gradient (as well as in estimation), it is more convenient to analyze the log-likelihood function than the likelihood function itself. Taking the natural logarithm of [equation] yields 
\begin{equation}
  SLL(\bm{\vartheta}) = \sum_{i=1}^N \log \Bigg( \frac{1}{R} \sum_{r=1}^R \prod_{t \in \mathcal{T}_i} \prod_{j \in \mathcal{J}_t}  \frac{1}{D} \sum_{d=1}^D \bigg( \varphi^{\text{order}}_{ijt, \, rd} 
  \cdot  \prod_{s \in \mathcal{J}_t \setminus \{j\} }\varphi^{\text{accept}}_{ist, \, rd} \bigg)  \Bigg), \label{eq:sim-loglik}
\end{equation}
where
\begin{equation*}
   \varphi^{\text{order}}_{ijt, \, rd} \coloneq \operatorname{Pr} [ \text{order}_{ijt} = 1 \mid \bm{\vartheta}_{it}^{rd} ]^{\text{order}_{ijt}}
\end{equation*}
and 
\begin{equation*}
   \varphi^{\text{accept}}_{ist, \, rd} \coloneq \operatorname{Pr} [\text{accept}_{ist} = 1 \mid \bm{\vartheta}_{it}^{rd}  ]^{\text{OOS}_{jt} \cdot \text{accept}_{ist}}.
\end{equation*}

Differentiating \cref{eq:sim-loglik} with respect to \(\bm{\vartheta}\), we obtain 
\begin{equation}
  \frac{\partial SLL(\bm{\vartheta})}{\partial \bm{\vartheta}} = \sum_{i=1}^N \frac{\frac{1}{R} \sum_{r=1}^R \frac{\partial}{\partial \bm{\vartheta}} \prod_{t=1}^T \Big( \frac{1}{R} \sum_{r=1}^R \big( \varphi^{\text{order}}_{ijt, \, rd} 
  \cdot  \prod_{s \in \mathcal{J}_t \setminus \{j\} }\varphi^{\text{accept}}_{ist, \, rd} \big) \Big)}{\frac{1}{R} \sum_{r=1}^R \prod_{t=1}^T \Big( \frac{1}{R} \sum_{r=1}^R \big( \varphi^{\text{order}}_{ijt, \, rd} 
  \cdot  \prod_{s \in \mathcal{J}_t \setminus \{j\} }\varphi^{\text{accept}}_{ist, \, rd} \big) \Big)} \label{eq:grad-init}
\end{equation}

Consider the derivative in the numerator of \cref{eq:grad-init}. Repeated application of the product rule yields
\begin{flalign}
  \frac{\partial}{\partial \bm{\vartheta}} \prod_{t=1}^T \Big( \frac{1}{R} \sum_{r=1}^R \big( \varphi^{\text{order}}_{ijt, \, rd} 
  \cdot  \prod_{s \in \mathcal{J}_t \setminus \{j\} }\varphi^{\text{accept}}_{ist, \, rd} \big) \Big) &= \bigg( \prod_{t=1}^T \Big( \frac{1}{R} \sum_{r=1}^R \big( \varphi^{\text{order}}_{ijt, \, rd} 
  \cdot  \prod_{s \in \mathcal{J}_t \setminus \{j\} }\varphi^{\text{accept}}_{ist, \, rd} \big) \Big) \bigg) \nonumber \\ 
  &\qquad \cdot \bigg( \sum_{t=1}^T \frac{\frac{1}{R} \sum_{r=1}^R \frac{\partial}{\partial \bm{\vartheta}} \big( \varphi^{\text{order}}_{ijt, \, rd} 
  \cdot  \prod_{s \in \mathcal{J}_t \setminus \{j\} }\varphi^{\text{accept}}_{ist, \, rd} \big)}{\frac{1}{R} \sum_{r=1}^R \big( \varphi^{\text{order}}_{ijt, \, rd} 
  \cdot  \prod_{s \in \mathcal{J}_t \setminus \{j\} }\varphi^{\text{accept}}_{ist, \, rd} \big)} \bigg) \nonumber \\
  &= \bigg( \prod_{t=1}^T \Big( \frac{1}{R} \sum_{r=1}^R \big( \varphi^{\text{order}}_{ijt, \, rd} 
  \cdot  \prod_{s \in \mathcal{J}_t \setminus \{j\} }\varphi^{\text{accept}}_{ist, \, rd} \big) \Big) \bigg) \nonumber \\ 
  &\qquad \cdot \Bigg( \sum_{t=1}^T \sum_{r=1}^R \bigg( \frac{\partial  \varphi^{\text{order}}_{ijt, \, rd} }{\partial \bm{\vartheta}} \cdot  \prod_{s \in \mathcal{J}_t \setminus \{j\} }\varphi^{\text{accept}}_{ist, \, rd} \nonumber \\ 
  &\qquad \qquad \qquad \qquad \quad + \varphi^{\text{order}}_{ijt, \, rd} 
  \cdot  \frac{\partial }{\partial \bm{\vartheta}} \Big(\prod_{s \in \mathcal{J}_t \setminus \{j\} }\varphi^{\text{accept}}_{ist, \, rd} \Big) \bigg) \Bigg) \nonumber \\
  &\qquad \big/  \sum_{r=1}^R \Big( \varphi^{\text{order}}_{ijt, \, rd} 
  \cdot  \prod_{s \in \mathcal{J}_t \setminus \{j\} }\varphi^{\text{accept}}_{ist, \, rd} \Big)
\end{flalign}
By Eqs. 28 and 29 of [Krueger year],
\begin{equation*}
  \frac{\partial \varphi^{\text{order}}_{ijt, \, rd}}{\partial \bm{\vartheta}} = \varphi^{\text{order}}_{ijt, \, rd} \cdot \frac{\partial V(w_{jt}; \bm{\vartheta}_{it})}{\partial \bm{\vartheta}_{it}}  - \sum_{j' \in \mathcal{J}_t \setminus \{j\}}  \varphi^{\text{order}}_{ijt, \, rd} \cdot \varphi^{\text{order}}_{ijt, \, rd} \cdot \frac{\partial V(w_{j't}; \bm{\vartheta}_{it})}{\partial \bm{\vartheta}_{it}}, 
\end{equation*}
where \ldots 

Likewise,
\begin{equation*}
  \frac{\partial \varphi^{\text{accept}}_{ist, \, rd} }{\partial \bm{\vartheta}} = \varphi^{\text{accept}}_{ist, \, rd}  \cdot \frac{\partial V(w_{jt}; \bm{\vartheta}_{it})}{\partial \bm{\vartheta}_{it}}  - \varphi^{\text{accept}}_{ist, \, rd} \cdot \varphi^{\text{reject}}_{ist, \, rd} \cdot \frac{\partial V(w_{0t}; \bm{\vartheta}_{it})}{\partial \bm{\vartheta}_{it}}, 
\end{equation*}
where 
\begin{equation*}
  \varphi^{\text{reject}}_{ist, \, rd} \coloneq \frac{1}{1 + \exp \big(  V(w_{st}; \bm{\vartheta}_{it}) \big)}.
\end{equation*}

\end{document}